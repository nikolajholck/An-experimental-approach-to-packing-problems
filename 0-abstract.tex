%\section*{Abstract}
\vspace*{\fill}
\begin{center}
{\scshape Abstract}
\end{center}
\par\medskip
\noindent This thesis employs experimental mathematics to examine a multi\-dimensional packing problem proposed by Dean G. Hoffman and based on the AM-GM-inequality. We formalize the problem and generalize several of Hoffman's results from the three-dimensional case to higher dimensions. We explain why it is complicated to count the number of unique packings and introduce the notion of a universal packing to remedy this. In the end, this enables us to provide the first rigorous proof that the three-dimensional case can always be solved.

We reproduce most of the results presented in the sparse literature on the subject, provide a rectified count of the number of unique squares in the four-dimensional case and give an estimate of the computations needed to determine the number of cubes in the four-dimensional case. We also construct a four-dimensional universal packing, formulate several hypotheses closely related to Maclaurin's inequality and prove one of these to be a necessary condition for being a universal packing. In addition, we use this universal packing as a stepping stone to give a lower bound on the number of unique four-dimensional universal packings.

Finally, we discuss how to approach the five-dimensional case. This includes ways of exploiting the hypotheses formulated during the investigation of the four-dimensional case and an assessment of the viability of the methods utilized to compute packings in lower dimensions.
\vspace*{2cm}
\vspace*{\fill}
